\documentclass[9pt]{beamer}
%\includeonlyframes{current}
\mode<article> % only for the article version                                                         
{                                                                                                
  \usepackage{fullpage}                                                                          
  \usepackage{hyperref}                                                                          
}                                                                                                     
\mode<presentation>                                                                                  
{   
  \setbeamertemplate{background canvas}[vertical shading][bottom=red!10,top=blue!10]  
  \setbeamercovered{transparent}
%  \usetheme{Boadilla}                                                                           
\usetheme{Goettingen}
%\usetheme{Marburg}
%\usetheme{Hannover}
%  \usetheme{Warsaw}
%    \usetheme{Dresden}                                                                          
    
  \usefonttheme[onlysmall]{structurebold} 
 \usecolortheme{seahorse}
%\usecolortheme{lily}                                                        
%\setbeamercovered{dynamic}
}

%%%%%%%%%%%%%%%%%%%%%%%%%%%%%%%%%%%%%%%%%%%%%%%%%%%%%%%%%%%%%%%%%%%

\usepackage{amssymb} 
\usepackage{amsmath} 
\usepackage{eufrak}     
\usepackage{eucal}                                                  
\usepackage{color} 
\usepackage{amsthm}  
\usepackage{mathrsfs}                                                           
\usepackage{slashbox}                                                           
\usepackage[all]{xy}                                                            
\usepackage{graphicx}                                                           
\usepackage{wrapfig}
%\usepackage{pgf,pgfarrows,pgfnodes,pgfautomata,pgfheaps,pgfshade}
\def\b{\bar}
\newcommand{\ud}{\mathrm{d}}
\newcommand{\p}{\partial}
\newcommand{\bp}{\bar{\partial}}
\def\cQ{\mathcal{Q}}
\def\cE{\mathcal{E}}
\def\bcE{\bar{\mathcal{E}}}
\def\cF{\mathcal{F}}
\def\N{\mathbb{N}}
\def\R{\mathbb{R}}
\def\a{\alpha}
\def\b{\bar}
\def\D{\Delta}
\def\n{\eta}
\def\inv{^{-1}}
\def\<{\langle}
\def\>{\rangle}
\def\+{\dagger}
\def\tab{\;\;\;\;\;\;}
\newcommand{\mat}[1]{\begin{pmatrix}#1\end{pmatrix}}

%%%%%%%%%%%%%%%%%%%%%%%%%%%%%%%%%%%%%%%%%%%%%%%%%%%%%%
\title[Einstein equations, Beltrami-Courant differentials and  Homotopy Gerstenhaber algebras]
{Einstein equations, Beltrami-Courant differentials and  Homotopy Gerstenhaber algebras}
\author[Anton Zeitlin]{Anton M. Zeitlin}
\institute[Columbia]{Columbia University, Department of Mathematics}
\date[2014]{{\tiny University of Pennsylvania}\\
{\tiny Philadelphia}\\
{\tiny December 11, 2015}}

\subject{Theoretical Physics & Mathematics}
%\pgfdeclareimage[height=0.7cm]{university-logo}{epdi-logo}
%\logo{\pgfuseimage{university-logo}}
\pgfdeclareimage[height=0.7cm]{university-logo}{ColumbiaU}
\logo{\pgfuseimage{university-logo}}
\pgfdeclareimage[height=0.7cm]{ihes-logo}{ihes-logo}
\logo{\pgfuseimage{ihes-logo}}

%%%%%%%%%%%%%%%%%%%%%%%%%%%
%%%%%%%%%%%%%%%%%%%%%%%%%%%%%%%%%%%%%%%%

\begin{document}

\section{}
\begin{frame}
\vspace*{10mm}
%\begin{flushleft}
%\pgfuseimage{university-logo}
%\end{flushleft}
%\transdissolve[duration=0.2]
\titlepage
\vspace*{10mm}
\begin{flushright}
\pgfuseimage{university-logo}
%\pgfuseimage{ihes-logo}
\end{flushright}
\end{frame}

%%%%%%%%%%%%%%%%%%%%%%%%%%%%%%%%%%%%%%%%%%%%
%%%%%%%%%%%%%%%%%%%%%%%%%%%%%%%%%%%%%%%%%%%%
%%%%%%%%%%%%%
%%%%%%%%%%%%%%%%%%%%%%%%%%%%%%%%%%%%%%%%%%%%
%%%%%%%%%%%%%%%%%%%%%%%%%%%%%%%%%%%%%%%%%%%%

\section*{Outline}
\begin{frame}[t]
%\transdissolve[duration=1]
\frametitle{Outline}
\tableofcontents
%\tableofcontents[pausesections, align=top] 
\end{frame}

%\AtBeginSubsection[]                                                                 
%{                                                                                    
%  \begin{frame}<beamer>                                                              
%    \frametitle{Outline}                                                             
%    \tableofcontents[current,currentsubsection]                                      
%  \end{frame}                                                                        
%}                 

%%%%%%%%%%%%%%%%%%%%%%%%%%%%%%%%%%%%%%%%%%%%%%%%%%%%%%%%%%%%%%%%%%%%%%%%%%%%%%%%%%
%%%%%%%%%%%%%%%%%%%%%%%%%%%%--------------1------------%%%%%%%%%%%%%%%%%%%%%%%%%%%
%\section[Superopers]{Superopers on Supercurves}
%%%%%%%%%%%%%%%%%%%%
%\begin{frame}[t]
%\frametitle{Superopers on Supercurves}

%\begin{itemize}
%\item[1.] Reminder: Opers and Gaudin models
%\item[2.] Supercurves and $osp(2|1)$ superopers
%\item[3.] Superopers for higher rank
%\item[4.] Superopers with regular singularity and Miura Superopers
%\item[5.] Relation to Gaudin model: $osp(2|1)$
%\end{itemize}

%\end{frame}
%%%%%%%%%%%%%%%%%%%%%%%%%-----------------2-------------%%%%%%%%%%%%%%%%%%%%%%%%%%
\section[Sigma-models and conformal invariance conditions]{Sigma-models and conformal invariance conditions}
\begin{frame}
\frametitle{Sigma-models and conformal invariance conditions}
Sigma-models for string theory in curved spacetimes: \\

\vspace*{3mm}

Let $X: \Sigma\to M$, where $\Sigma$ is a compact Riemann surface (worldsheet) and  $M$ is a Riemannian manifold (target space).\\

\vspace{3mm}

Action functional of sigma model:
\begin{eqnarray*}
S_{so}=\frac{1}{4\pi h}\int_{\Sigma}(G_{\mu\nu}(X)dX^\mu\wedge *dX^{\nu}+X^*B)
\end{eqnarray*}
where $G$ is a metric on $M$, $B$ is a 2-form on $M$.\\

\vspace*{3mm}

Symetries: \\

\vspace*{1mm}

i) conformal symmetry on the worldsheet, \\  

\vspace*{1mm}

ii) diffeomorphism symmetry and $B\to B+d\lambda$ on target space. \\

\end{frame}
\begin{frame}

  
On the quantum level one can add one more term to the action: 
\begin{eqnarray*}
S_{so}\to S_{so}^{\Phi}=S_{so}+ \int_{\Sigma}\Phi(X)R^{(2)}(\gamma){\rm vol}_{\Sigma},
\end{eqnarray*}
where function $\Phi$ is called {\it dilaton}, $\gamma$ is a metric on $\Sigma$.\\

\vspace*{3mm}

In order to make sense of path integral
\begin{eqnarray*}
Z=\int DX ~\ e^{-S^{\Phi}_{so}(X, \gamma)}
\end{eqnarray*}
one has to apply renormalization procedure, so that $G$, $B$, $\Phi$ depend 
on certain $cutoff$ parameter $\mu$, so that in general quantum theory is not conformally invariant. 

\end{frame}
\begin{frame}

Conformal invariance conditions are:
\begin{eqnarray*}
&&\mu\frac{d}{d\mu}G_{\mu\nu}=\beta^G_{\mu\nu}(G,B,\Phi, h)=0, \quad  \mu\frac{d}{d\mu}B_{\mu\nu}=\beta^B_{\mu\nu}(G,B,\Phi, h)=0,\\ 
&&\mu\frac{d}{d\mu}\Phi=\beta^{\Phi}(G,B,\Phi, h)=0
\end{eqnarray*}

at the level $h^0$ turn out to be Einstein Equations with $Kalb-Ramond$ 2-form field $B$ and dilaton $\Phi$:
\begin{eqnarray*}
&&R_{\mu\nu}={1\over 4} H_{\mu}^{\lambda\rho}H_{\nu\lambda\rho}-2\nabla_{\mu}
\nabla_{\nu}\Phi,\\
&&\nabla^{\mu}H_{\mu\nu\rho}-2(\nabla^{\lambda}\Phi)H_{\lambda\nu\rho}=0,
\nonumber\\
&&4(\nabla_{\mu}\Phi)^2-4\nabla_{\mu}\nabla^{\mu}\Phi+
R+{1\over 12} H_{\mu\nu\rho}H^{\mu\nu\rho}=0,\nonumber
\end{eqnarray*}
where 3-form $H=dB$, and $R_{\mu\nu}, R$ are Ricci and scalar curvature correspondingly. 
\end{frame}

%%%%%%%%%%%%%%%%%%%%%%%-------------3-------------%%%%%%%%%%%%%%%%%%%%%%%%%%%%%
%\section[Gaudin model]{Gaudin model}
\begin{frame}
\frametitle{}
In the early days of string theory:\\

\vspace*{3mm}

Linearized Einstein Equations and their symmetries:\\ 
($G_{\mu\nu}=\eta_{\mu\nu}+s_{\mu\nu}$, $B_{\mu\nu}=b_{\mu\nu}$, $\Phi=\phi$):
\begin{eqnarray*}
Q^{\eta}\Psi(s, b, \phi)=0, \quad \Psi^s(s, b, \phi)\to \Psi(s, b, \phi)+Q^{\eta}\Lambda
\end{eqnarray*}
in a semi-infinite complex associated to Virasoro module of Hilbert space of states for the "free" theory, asssociated to flat metric. \\

\vspace*{3mm}

It was conjectured (A. Sen, B. Zwiebach,...) in the early 90s that Einstein equations with $h$-corrections are Generalized Maurer-Cartan (GMC) Equations:
\begin{eqnarray*}
Q^{\eta}\Psi+
\frac{1}{2}[\Psi, \Psi]_h+\frac{1}{3!}[\Psi, \Psi, \Psi]_h+...=0
\end{eqnarray*}
\begin{eqnarray*}
\Psi\to \Psi+Q^{\eta}\Lambda+[\Psi,\Lambda]_h+\frac{1}{2}[\Psi,\Psi,\Lambda]_h+...,
\end{eqnarray*}

where $[\cdot, \cdot, ..., \cdot]_h$ operations, together with differential 
$Q$ satisfy certain bilinear relations and generate $L_{\infty}$-algebra ($L$ stands for Lie).
\end{frame}

\begin{frame}

In this talk:\\

\vspace*{3mm}

i) Introducing complex structure: \\
\vspace*{1mm}
Proper chiral "free action" $\to$  
sheaves of vertex algebras/vertex algebroids.\\ 

\vspace*{1mm} 
 
Metric, $B$-field $\to$ Beltrami-Courant differential.

\vspace*{3mm}

ii) Vertex algebroids $\to$ $G_{\infty}$-algebras ($G$ stands for Gerstenhaber). \\

\vspace*{1mm}
  
Quasiclassical limit:\\ 
vertex algebroid $\to$ Courant algebroid, $G_{\infty}$ algebra is truncated.

\vspace*{3mm}

iii) Einstein equations and their $h$-corrections via Generalized Maurer-Cartan equation for $L_{\infty}$-subalgebra of $G_{\infty}\otimes \bar{G}_{\infty}$.



\end{frame}

%%%%%%%%%%%%%%%%%%%%---------------4-------------------%%%%%%%%%%%%%%%%%%%%%%%%%%


\begin{frame}
\frametitle{First order version of sigma-model action}
\section[Beltrami-Courant differential]{Beltrami-Courant differential and  first order sigma-models}
We start from the action functional:
\begin{eqnarray*}
&&S_0=\frac{1}{2\pi ih}\int_\Sigma \mathcal{L}_0, \quad \mathcal{L}_0=\langle p\wedge\bar{\partial} X\rangle-
\langle \bar{p}\wedge{\partial} X\rangle,
\end{eqnarray*}
where $p$, $\b p$ are sections of 
$X^*(\Omega^{(1,0)}(M))\otimes \Omega^{(1,0)}(\Sigma)$, $X^*(\Omega^{(0,1)}(M))\otimes \Omega^{(0,1)}(\Sigma)$ correspondingly.\\\vspace*{3mm}

Infinitesimal local symmetries:
\begin{eqnarray*}
\mathcal{L}_0\to \mathcal{L}_0+d\xi
\end{eqnarray*}

For holomorphic transformations we have:
\begin{eqnarray*}
&& X^i\to X^i-v^i(X), X^{\b i}\to X^{\b i}-v^{\b i}(\b X),\\
&&p_i\to p_i+\p_i v^k p_k, \quad p_{\b i}\to p_{\b i}+\p_{\b i} v^{\b k} p_{\b k}\\
&&p_i\to p_i-\p X^k(\p_k\omega_i-\p_i\omega_k), \quad p_{\b i}\to p_{\b i}-\bar{\p} X^{\b k}(\p_{\b k}\omega_{\b i}-\p_{\b i}\omega_{\b k}).
\end{eqnarray*}
Not invariant under general diffeomorphisms, i.e. 
\begin{eqnarray*}
 \delta \mathcal{L}_0=-\langle \b \p v, p\wedge \b \p X\rangle +\langle \p \b v , \b p\wedge \p X\rangle.
\end{eqnarray*}




\end{frame}
%%%%%%%%%%%%%%%%%%%%%%%%%--------5------%%%%%%%%%%%%%%%%%%%%%%%%%%%%%%%%%%%%%%%%%%%%%%%


\begin{frame}
\frametitle{}
It is necessary to add extra terms:
\begin{eqnarray*}
\delta \mathcal{L}_{\mu}=- \langle \mu, p\wedge \b \p X\rangle -\langle \b \mu, \p X\wedge \b p\rangle,
\end{eqnarray*}
where $\mu\in \Gamma (T^{(1,0)}M\otimes {T^*}^{(0,1)}(M))$, $\bar{\mu}\in \Gamma (T^{(0,1)}M\otimes {T^*}^{(1,0)}(M))$, so that:
$\mu\to \mu-\b \p v+\dots$, $\b \mu\to \b \mu-\p \b v+\dots$.\\

\vspace*{2mm}

Continuing the procedure:
\begin{eqnarray*}
&&\tilde{\mathcal{L}}=
\langle p\wedge\bar{\partial} X\rangle-
\langle \bar{p}\wedge{\partial} X\rangle -\\
&&\langle \mu, p\wedge \b \p X\rangle -\langle \b \mu, \p X\wedge \b p\rangle-\langle b, \p X\wedge \b \p X\rangle,\nonumber
\end{eqnarray*}
where
\begin{eqnarray*}
 &&\mu^{i}_{\bar{j}} \rightarrow \\
 &&\mu^{i}_{\bar{j}} -
\p_{\bar{j}}v^i + v^{k}\p_k\mu^{i}_{\bar{j}} +
v^{\bar{k}}\p_{\bar{k}}\mu^{i}_{\bar{j}}+
\mu^{i}_{\bar{k}}\p_{\bar{j}}v^{\bar{k}} -
\mu^k_{\bar{j}}\p_kv^i
 + \mu^i_{\bar{l}}\mu^k_{\bar{j}}\p_k v^{\bar{l}},\\ 
&&b_{i{\bar j}} \rightarrow \\
&&b_{i{\bar j}} + v^k\p_k b_{i{\bar j}} + v^{\bar{k}}\p_{\bar{k}} b_{i{\bar j}}
+ b_{i{\bar k}}\p_{\bar{j}}v^{\bar{k}}+b_{l{\bar j}}\p_i v^l+b_{i{\bar k}}\mu^{k}_{\bar j}\p_kv^{\bar{k}}
+b_{l{\bar j}}{\bar\mu}^{\bar k}_i\p_{\bar k}v^l,\nonumber
 \end{eqnarray*}
so that the transformations of $X$- and $p$- fields are:
\begin{eqnarray*}
&&X^i\to X^i-v^i(X, \b X), \quad p_{i} \rightarrow p_{i} + p_k \p_iv^k - p_k\mu^k_{\bar{l}}\p_iv^{\bar{l}}
- b_{j{\bar k}}\p_i v^{\bar k}\p X^j,\\
&& X^{\b i}\to X^{\b i}-v^{\b i}(X,\b X), \quad \bar{p}_{\bar{i}} \rightarrow \bar{p}_{\bar{i}} + \bar{p}_{\bar k} \p_{\bar i}v^{\bar {k}} - {\bar p}_{\bar{k}}
{\bar \mu}^{\bar k}_{l}\p_iv^{l}
- b_{\bar{j}k}\p_{\bar{i}} v^{k}\bar{\p} X^{\bar {j}}.\nonumber
\end{eqnarray*}


\end{frame}

%%%%%%%%%%%%%%%%%%%%%%%%%--------6---------%%%%%%%%%%%%%%%%%%%%%%%%%%%%%%%%%%%%%%%%%%%%%%

\begin{frame}
Similarly, for the 1-form transformation we obtain:

\begin{eqnarray*}
&&b_{i\bar{j}}\to  b_{i\bar{j}}+\p_{\bar j}\omega_i-\p_i\omega_{\bar j}+\mu^i_{\bar j}(\p_i\omega_k-\p_k\omega_i)+\nonumber\\
&&\bar{\mu}^{\bar s} _i(\p_{\bar j}\omega_{\bar s}-
\p_{\bar s}\omega_{\bar j})+
{\bar \mu}^{\bar i}_j\mu_{\bar k}^s(\p_s\omega_{\bar i}-\p_{\bar i}\omega_s)
\end{eqnarray*}

and

\begin{eqnarray*}
&& p_i\to p_i-\p X^k(\p_k\omega_i-\p_i\omega_k)-\p_{\b r}\omega_i\p X^{\bar r}-{\bar \mu}^{\bar s}_k\p_i\omega_{\bar s}\p X^k, \nonumber\\
&& p_{\b i}\to p_{\b i}-{\bar\p} X^{\b k}(\p_{\b k}\omega_{\b i}-{\p}_{\b i}\omega_{\b k})-\p_{r}\omega_{\bar i}{\bar\p} X^{r}-{\mu}^{s}_{\bar k}\p_i\omega_{s}{\bar\p} X^{\bar k}.
\end{eqnarray*}

\vspace*{5mm}

For simplicity:  
\begin{eqnarray*}
&&E=TM\oplus T^*M,  \quad E=\mathcal{E}\oplus\bar{\mathcal{E}},\\
&&\mathcal{E}=T^{(1,0)}M\oplus {T^*}^{(1,0)}M, \quad \bar{\mathcal{E}}=T^{(0,1)}M\oplus {T^*}^{(0,1)}M.
\end{eqnarray*}



\end{frame}

%%%%%%%%%%%%%%%%%%%%%%%%%%----------7------------%%%%%%%%%%%%%%%%%%%%%%%%%%%%%%%%%%%%%%%
%\section[$osp(2|1)$ superoper]{$osp(2|1)$ superoper}
\begin{frame}[t]


Let $\tilde{\mathbb{M}}\in \Gamma(\mathcal{E}\otimes\bar{\mathcal{E}})$, such that
\begin{equation*}
\tilde{\mathbb{M}}=\begin{pmatrix} 0 & \mu \\ 
\bar{\mu} & b \end{pmatrix}.
\end{equation*}
Introduce $\alpha\in \Gamma(E)$, i.e. $\alpha=(v, \bar v, \omega, \bar \omega)$. 
Let $D:\Gamma(E)\to \Gamma(\cE\otimes\bcE)$, such that

\[ D\alpha=\left( \begin{array}{cc}
0 & {\bar \p }v\\
{\p \bar v} & \p{\bar\omega}-{\bar\p} \omega \end{array} \right).\]

Then the transformation of $\tilde{\mathbb{M}}$  can be expressed:
\begin{eqnarray*}\label{algsym}
\tilde{\mathbb{M}}\to \tilde{\mathbb{M}}-D\alpha+ \phi_1(\alpha,\tilde{\mathbb{M}})+\phi_2(\alpha, \tilde{\mathbb{M}},\tilde{\mathbb{M}}).
\end{eqnarray*}

Let us describe $\phi_1, \phi_2$ algebraically. In order to do that we need to pass to jet bundles, i.e. 
$$\alpha\in J^{\infty}(\mathcal{O}_M)\otimes 
J^{\infty}({\bar{\mathcal{O}}}(\bcE))
\oplus J^{\infty}({\mathcal{O}}(\cE))
\otimes J^{\infty}({\bar{\mathcal{O}}}_M),$$
 
$$\tilde{\mathbb{M}}\in J^{\infty}({\mathcal{O}}(\cE))\otimes J^{\infty}({\bar{\mathcal{O}}}(\bcE))$$

\end{frame}


%%%%%%%%%%%%%%%%%%%%%%%%%%---------8---------%%%%%%%%%%%%%%%%%%%%%%%%%%%%%%%%%%%%%%%%%%%%%%

\begin{frame}[t]

Then:
\begin{eqnarray*}
&&\alpha=\sum_Jf^J\otimes {\bar{b}}^J+\sum_Kb^K\otimes {\bar{f}}^K,\nonumber\\ 
&&\tilde{\mathbb{M}}=\sum_I a^I\otimes \bar{a}^I, 
\end{eqnarray*}
where $a^I, b^J\in J^{\infty}({\mathcal{O}}(\cE))$, $f^I\in 
J^{\infty}(\mathcal{O}_M)$ and ${\bar{a}}^I, {\bar{b}}^J\in J^{\infty}({\bar{\mathcal{O}}}(\bcE))$, 
${\bar{f}}^I\in 
J^{\infty}(\bar{\mathcal{O}}_M)$. Then 
\begin{eqnarray*}
\phi_1(\alpha,\tilde{\mathbb{M}})=
\sum_{I,J}[b^J, a^I]_D\otimes 
{\bar{f}}^{J}{\bar{a}}^I+\sum_{I,K}f^Ka^I\otimes[{\bar{b}}^K, {\bar{a}}^I]_D,
\end{eqnarray*}
where $[\cdot, \cdot]_D$ is a $Dorfman$ $bracket$:
\begin{eqnarray*}
&&[v_1,v_2]_D=[v_1,v_2]^{Lie}, \quad [v,\omega]_D=L_v\omega,\nonumber\\
&&[\omega , v]_D=-i_vd\omega,\quad [\omega_1,\omega_2]_D=0.
\end{eqnarray*}
Courant bracket is the antysymmetrized version of $[\cdot,\cdot]_D$.\\


Similarly:
\begin{eqnarray*}
&&\phi_2(\alpha, \tilde{\mathbb{M}},\tilde{\mathbb{M}})=\tilde{\mathbb{M}}\cdot D\alpha\cdot \tilde{\mathbb{M}}
\nonumber\\
&&\frac{1}{2}\sum_{I,J,K}\langle b^I, a^K\rangle a^J\otimes \bar{a}^J(\bar{f}^I) {\bar{a}}^K+\frac{1}{2}\sum_{I,J,K}
 a^J(f^I) {a}^K\otimes \langle {\bar{b}}^I, {\bar{a}}^K\rangle {\bar{a}}^J.
\end{eqnarray*}


\end{frame}



%%%%%%%%%%%%%%%%%%%%%%%%%%---------9----------%%%%%%%%%%%%%%%%%%%%%%%%%%%%%%%%%%%%%%%%%%%%
\begin{frame}[t]
Relation to standard second order sigma-model: Let us fill in $0$ in $\tilde{\mathbb{M}}$:
\begin{equation*}\label{mmat}
\mathbb{M}=\begin{pmatrix} g & \mu \\ 
\bar{\mu} & b \end{pmatrix}.
\end{equation*}
\begin{eqnarray*}
&&S_{fo}=\frac{1}{2\pi i h}\int_\Sigma (\langle p\wedge\bar{\partial} X\rangle-
\langle \bar{p}\wedge{\partial} X\rangle-\nonumber\\
&& -\langle g, p\wedge \bar{p} \rangle-\langle \mu, p\wedge \b \p X\rangle -\langle \b \mu, \b p\wedge \p X\rangle-\langle b, \p X\wedge \b \p X\rangle).
\end{eqnarray*}
\hfill{\color{blue}\tiny V.N. Popov, M.G. Zeitlin, Phys.Lett. B 163 (1985) 185, A. Losev, A. Marshakov, A.Z., Phys. Lett. B 633 (2006) 375}\\
Same formulas express symmetries. If $\{g^{i\b j} \}$ is nondegenerate, then :
\begin{eqnarray*}
&&S_{so}=\frac{1}{4\pi h}\int_{\Sigma}(G_{\mu\nu}(X)dX^{\mu}\wedge *dX^{\nu}+X^*B),\\
G_{s\bar{k}}&=&g_{\bar{i}j}
\bar{\mu}^{\bar{i}}_s\mu^{j}_{\bar{k}}+g_{s\bar{k}}-
b_{s\bar{k}}, \quad
B_{s\bar{k}}=g_{\bar{i}j}\bar{\mu}^{\bar{i}}_s\mu^{j}_{\bar{k}}-g_{s\bar{k}}-
b_{s\bar{k}}\\
G_{si}&=&-g_{i\bar{j}}\bar{\mu}^{\bar{j}}_s-g_{s\bar{j}}\bar{\mu}^{\bar{j}}_i
, \quad
G_{\bar{s}\bar{i}}=-g_{\bar{s}j}\mu^{j}_{\bar{i}}-g_{\bar{i}j}\mu^{j}_{\bar{s}}
\nonumber\\
B_{si}&=&g_{s\bar{j}}\bar{\mu}^{\bar{j}}_i-g_{i\bar{j}}\bar{\mu}^{\bar{j}}_s,
\quad
B_{\bar{s}\bar{i}}=g_{\bar{i}j}\mu^{j}_{\bar{s}}-g_{\bar{s}j}\mu^{j}_{\bar{i}}.
\end{eqnarray*}
Symmetries $\mathbb{M}\to \mathbb{M}-D\alpha+ \phi_1(\alpha,{\mathbb{M}})+\phi_2(\alpha, {\mathbb{M}},{\mathbb{M}})$ are equivalent to:\\ 
\hfill{\color{blue}\tiny  A.Z., Adv. Theor. Math. Phys. (2015), to appear}
\begin{eqnarray*}
&& G\to G-L_{\bf v}G,\quad B\to B-L_{\bf v}B\\ 
&& B\to B-2d{\bf\boldsymbol \omega}\nonumber\\
&& \alpha=({\bf v}, {\boldsymbol \omega}), \quad {\bf v}\in \Gamma(TM), {\boldsymbol \omega}\in \Omega^{1}(M)
\end{eqnarray*}
\end{frame}

%%%%%%%%%%%%%%%%%%%%%%%%%%------------10-----------%%%%%%%%%%%%%%%%%%%%%%%%%%%%%%%%%%%
\section{Vertex/Courant algebroids, $G_{\infty}$-algebra and quasiclassical limit}
\begin{frame}
\frametitle{Vertex algebroids}
The quantum theory, corresponding to the chiral part of the free first order Lagrangian $\mathcal{L}_0$ is described (under certain constraints on $M$) via sheaves of VOA on $M$ (V. Gorbounov, F. Malikov, V. Schechtman, A. Vaintrob).  \\

On the open set $U$ of $M$ we have VOA:
\begin{eqnarray*}
V=\sum_{n=0}^{\infty}V_n, \quad  Y: V\to End(V)[[z,z^{-1}]], 
\end{eqnarray*}
generated by:
\begin{eqnarray*}
&&[X^i(z),p_j(w)]=h\delta^i_j\delta(z-w), \quad i,j=1,2,\dots ,D/2\\
&&X^i(z)=\sum_{r\in \mathbb{Z}}X^i_rz^{-r}, p_j(z)=\sum_{s\in \mathbb{Z}} p_{j,s}z^{-s-1}\in End(V)[[z,z^{-1}]].
\end{eqnarray*}
so that 
\begin{eqnarray*}
&&V={\rm Span}\{p_{j_1, -s_1}, \dots, p_{j_k, -s_k}X^{i_1}_{-r_1}\dots 
X^{i_l}_{-r_l}\}\otimes F(U)\otimes{\mathbb{C}[h,h^{-1}]}, \\
&& r_m, s_n > 0,
\end{eqnarray*}
$F(U)$ generated by $X^i_0$-modes.

\end{frame}
\begin{frame}
The Virasoro element is:
\begin{eqnarray*}
T(z)=\sum_{n\in \mathbb{Z}} L_n z^{-n-2}=\frac{1}{h}:\langle p(z)\p X(z)\rangle:+\p^2\phi'(X(z)).
\end{eqnarray*}
\begin{eqnarray*}
[L_n,L_m]=(n-m)L_{n+m}+\frac{D}{12}(n^3-n)\delta_{n,-m}
\end{eqnarray*}
 corresponding to correction:
\begin{eqnarray*}
\mathcal{L}_0\to \mathcal{L}_{\phi'}=\langle p\wedge\bar{\partial} X\rangle-2\pi i h R^{(2)}(\gamma)\phi'(X)
\end{eqnarray*}
where $\phi'=\log\Omega$, where $\Omega(X)dX^1\wedge \dots\wedge dX^n$ is a holomorphic volume form, i.e. for globally defined $T(z)$, $M$ has to 
be Calabi-Yau.

The space $V$ is a lowest weight module for the above Virasoro algebra. \\

\vspace*{2mm}

$V$ can be reproduced from $V_0$ and $V_1$ as a $vertex$ $envelope$.
The structure of vertex algebra imposes algebraic relations on $V_0\oplus V_1$ giving it a structure of a $vertex$ $algebroid$.

\vspace*{2mm}

In our case:
$V_0\to  \mathcal{O}_M^h=\mathcal{O}_M\otimes \mathbb{C}[h,h^{-1}]$,\\  
\hspace*{17mm}$V_1\to \mathcal{V}^h=\mathcal{V}\otimes \mathbb{C}[h, h^{-1}],\quad  \mathcal{V}=\mathcal{O}(\cE)$
\end{frame}
\begin{frame}


A {\em vertex $\mathcal{O}_M$-algebroid} is a sheaf of $\mathbb{C}$-vector 
spaces $\mathcal{V}$ with a\\

\vspace*{1mm}

i) $\mathbb{C}$-linear pairing $\mathcal{O}_M\otimes\mathcal{V}  \to  \mathcal{V}[h]$, i.e.  
$f\otimes v  \mapsto  f*v$ 
such that $1* v = v$.\\

\vspace*{1mm}

ii) $\mathbb{C}$-linear bracket, satisfying  Leibniz algebra 
$[\ ,\ ] :
\mathcal{V}\otimes\mathcal{V}\to h\mathcal{V}[h]$,\\

\vspace*{1mm}

iii)$\mathbb{C}$-linear map of Leibniz algebras $\pi : \mathcal{V}\to h\Gamma({TM})[h]$ usually referred to as an anchor\\ 

\vspace*{1mm}

iv) a symmetric $\mathbb{C}$-bilinear pairing $\langle\ ,\ \rangle :
\mathcal{V}\otimes \mathcal{V}\to h\mathcal{O}_M[h]$,

\vspace*{1mm}

v) a $\mathbb{C}$-linear map $\p : \mathcal{O}_M\to \mathcal{V}$ 
such that
$\pi\circ\partial = 0$,\\

\vspace*{1mm}

naturally extending to $\mathcal{O}^h_M$ and $\mathcal{V}^h$, and satisfy the relations
\begin{eqnarray*}
&& f*(g*v) - (fg)*v  =  \pi(v)(f)*\partial(g) +
\pi(v)(g)*\partial(f),\nonumber\\
&&[v_1,f*v_2]  =  \pi(v_1)(f)*v_2 + f*[v_1,v_2], 
\nonumber\\
&&[v_1,v_2] + [v_2,v_1]  =  \partial\langle v_1,v_2\rangle,
\quad
\pi(f*v) = f\pi(v),  \nonumber\\
&&\langle f*v_1, v_2\rangle  =  f\langle v_1,v_2\rangle -
\pi(v_1)(\pi(v_2)(f)), \nonumber\\
&&\pi(v)(\langle v_1, v_2\rangle)  =  \langle[v,v_1],v_2\rangle +
\langle v_1,[v,v_2]\rangle,\nonumber \\
&&\partial(fg)  =  f*\partial(g) + g*\partial(f), \nonumber\\
&&[v,\partial(f)] =  \partial(\pi(v)(f)), \quad
\langle v,\partial(f)\rangle  =  \pi(v)(f),
\end{eqnarray*}
where $v,v_1,v_2\in\mathcal{V}^h$, $f,g\in\mathcal{O}_M^h$. \\

\end{frame} 




%%%%%%%%%%%%%%%%%%%%%%%%%%%%%%%%%%------11----------%%%%%%%%%%%%%%%%%%%%%%%%%%%%%%%
\begin{frame}[t]

For our considerations $\mathcal{V}=\mathcal{O}(\mathcal{E})$:
\begin{eqnarray*}
&&\p f=df,\quad \pi(v)f=-hv(f), \quad\pi(\omega)=0,\nonumber\\
&&f*v=fv+hdX^i\p_i\p_jfv^j, \quad f*\omega=f\omega,\nonumber\\
&&[v_1,v_2]=-h[v_1,v_2]_D-h^2dX^i\p_i\p_kv^s_1\p_sv^k_2,\nonumber\\
&&[v,\omega]=-h[v,\omega]_D, \quad 
[\omega, v]=-h[\omega,v]_D, \quad [\omega_1,\omega_2]=0,\nonumber\\  
&&\langle v, \omega\rangle=-h\langle v, \omega\rangle^s, \quad \langle v_1, v_2\rangle=-h^2\p_iv_1^j\p_jv_2^i,\quad \langle\omega_1. \omega_2\rangle=0,
\end{eqnarray*}
where v and $\omega$ are vector fields and 1-forms correspondingly. \\

\vspace*{2mm}

Together with ${\rm div}_{\phi'}$-the divergence operator with respect to $\phi'$ these operations generate vertex algebroid with Calabi-Yau structure.

\end{frame}

\begin{frame}

Vertex algebra $V$ is a Virasoro module. The corresponding semi-infinite complex $V^{semi}$ (the analogue of Chevalley complex for Virasoro algebra) is a vertex algebra too: 
\begin{eqnarray*}
&&V^{semi}=V\otimes \Lambda, \\
&&\Lambda\quad {\rm generated~\ by} \quad [b(z), c(w)]_+=\delta(z-w).
\end{eqnarray*}
The corresponding differential  
\begin{eqnarray*}
Q=j_0, \quad j(z)=\sum_{n\in\mathbb{Z}}j_nz^{-n-1}= c(z)T(z) + :c(z)\p c(z)b(z): 
\end{eqnarray*}
is nilpotent when $D=26$ (famous dimension 26!).  
However, we will consider subcomplex of light modes  (i.e. $L_0=0$) denoted in the following as $(\mathcal{F}_h^{\cdot}, Q)$, where we can drop this condition:
\begin{eqnarray*}
\xymatrixcolsep{30pt}
\xymatrixrowsep{3pt}
\xymatrix{
& \mathcal{V}^h\ar[ddddr] & \mathcal{V}^h\ar[ddddr]^{\frac{1}{2}h{\rm div}}& \\
&& \p &&\\
& \bigoplus & \bigoplus & \\
&& {-\frac{1}{2}h{\rm div}} &&\\
\mathcal{O}_M^{h}\ar[uuuur]^{\p} & \mathcal{O}_M^{h}\ar[uuuur]\ar[r]_{i\rm{d}} & \mathcal{O}_M^{h}  & \mathcal{O}_M^{h}.
}
\end{eqnarray*}



\end{frame}
%%%%%%%%%%%%%%%%%%%%%%%%%%%%%%%%--------12-----------%%%%%%%%%%%%%%%%%%%%%%%%%%%%%%

\begin{frame}[t]
\frametitle{The homotopy Gerstenhaber algebra of Lian and Zuckerman}
The homotopy associative and homotopy commutative product of Lian and Zuckerman:
\begin{eqnarray*}
(A,B)=Res_z\frac{A(z)B}{z}
\end{eqnarray*}

\begin{eqnarray*}&&Q(a_1,a_2)_h=(Q a_1,a_2)_h+(-1)^{|a_1|}(a_1,Q a_2)_h,\nonumber\\
&&(a_1,a_2)_h-(-1)^{|a_1||a_2|}(a_2,a_1)_h=\nonumber\\
&&Qm(a_1,a_2)+m(Qa_1,a_2)+(-1)^{|a_1|}m(a_1,Qa_2),\nonumber\\
&& Q(a_1,a_2,a_3)_h+(Qa_1,a_2,a_3)_h+(-1)^{|a_1|}(a_1,Qa_2,a_3)_h+\nonumber\\
&&(-1)^{|a_1|+|a_2|}(a_1,a_2,Qa_3)_h=((a_1,a_2)_h,a_3)_h-(a_1,(a_2,a_3)_h)_h\nonumber\\
\end{eqnarray*}

Operator $\mathbf{b}$ of degree -1 (0-mode of $b(z)$) on 
$(\mathcal{F}_h^{\cdot}, Q)$ which anticommutes with $Q$:  
\begin{eqnarray*}
\xymatrixcolsep{30pt}
\xymatrixrowsep{3pt}
\xymatrix{
& \mathcal{V}^h & \mathcal{V}^h\ar[l]_{-i\rm{d}}& \\
& \bigoplus & \bigoplus & \\
\mathcal{O}_M^{h} & \mathcal{O}_M^{h}\ar[l]_{i\rm{d}} & \mathcal{O}_M^{h}& \mathcal{O}_M^{h}\ar[l]_{-i\rm{d}}
} 
\end{eqnarray*}




\end{frame}

%%%%%%%%%%%%%%%%%%%%%%%%%%%%%-----------13------------%%%%%%%%%%%%%%%%%%%%%%%%%%%%%%%%%

\begin{frame}[t]
One can define a bracket:

\begin{eqnarray*}\label{brack}
(-1)^{|a_1|}\{a_1,a_2\}_h=\mathbf{b}(a_1,a_2)_h-(\mathbf{b}a_1,a_2)_h-(-1)^{|a_1|}(a_1\mathbf{b}a_2)_h,
\end{eqnarray*}

\begin{eqnarray*}
&&\{a_1,a_2\}_h+(-1)^{(|a_1|-1)(|a_2|-1)}\{a_2,a_1\}_h=\nonumber\\
&&(-1)^{|a_1|-1}(Qm_h'(a_1,a_2)-m_h'(Qa_1,a_2)-(-1)^{|a_2|}m_h'(a_1,Qa_2)),
\nonumber\\
&& \{a_1,(a_2,a_3)_h\}_h=(\{a_1,a_2\}_h,a_3)_h+(-1)^{(|a_1|-1)||a_2|}(a_2,\{a_1, a_3\}_h)_h,\nonumber\\
&&\{(a_1,a_2)_h,a_3\}_h-(a_1,\{a_2,a_3\}_h)_h-(-1)^{(|a_3|-1)|a_2|}(\{a_1,a_3\}_h,a_2)_h=\nonumber\\
&&(-1)^{|a_1|+|a_2|-1}(Qn_h'(a_1,a_2,a_3)-n_h'(Qa_1,a_2,a_3)-\nonumber\\
&&(-1)^{|a_1|}n_h'(a_1,Qa_2,a_3)-(-1)^{|a_1|+|a_2|}n_h'(a_1,a_2,Qa_3),\nonumber\\
&&\{\{a_1,a_2\}_h,a_3\}_h-\{a_1,\{a_2,a_3\}_h\}_h+\nonumber\\
&&(-1)^{(|a_1|-1)(|a_2|-1)}\{a_2,\{a_1,a_3\}_h\}_h=0.\nonumber
\end{eqnarray*}
The conjecture of Lian and Zuckerman, which was later proven by series of papers (Kimura, Zuckerman, Voronov; Huang, Zhao; Voronov) says that the symmetrized product and bracket of homotopy Gerstenhaber algebra constructed above can be lifted to $G_{\infty}$-algebra.


\end{frame}
%%%%%%%%%%%%%%%%%%%%%%%%%%%%-----------14--------------%%%%%%%%%%%%%%%%%%%%%%%%%%%%%%%%%

\begin{frame}[t]
\frametitle{Homotopy algebras: $G_{\infty}$, $L_{\infty}$, $C_{\infty}$}
Let $A$ be a graded vector space, consider free graded Lie algebra $Lie (A)$. 
\begin{eqnarray*}
Lie^{k+1}(A)=[A,Lie^{k}A], \quad Lie^{1}(A)=A.
\end{eqnarray*}
Consider free graded commutative algebra $GA$ on the suspension $(Lie(A))[-1]$, i.e. 
\begin{eqnarray*}
GA=\oplus_n{\bigwedge}^n Lie (A)[-n]
\end{eqnarray*}
There are natural $[\cdot, \cdot]$, $\wedge$ 
operations on $GA$ of degree -1, 0 correpondingly, generating a Gerstenhaber algebra.\\

\vspace*{3mm} 

A $G_{\infty}$-$algebra$ (Tamarkin, Tsygan, 2000) is a graded space $V$ with a differential $\partial$ of degree 1 of $G(V[1]^*)$, such that $\p$ is a derivation w.r.t bracket and the product.\\

\vspace*{3mm}

Multiplicative Ideals, preserved by $\p$: $I_1$-the commutant of $Lie(V[1]^*)$, $I_2=\bigwedge_{n\ge 2} (Lie(V[1]^*)[-n]$. That induces differentials on corresponding factors: $\bigwedge_{n\ge 1}(V[1]^*)[-n]$ and $Lie(V[1]^*)[-1]$. The resulting structures on $V$ are called $L_{\infty}$-$algebra$ and $C_{\infty}$-$algebra$ correspondingly.

\end{frame}

\begin{frame}[t]
Restriction of $\p$ on $V[1]^{*}$:
\begin{eqnarray*}
V[1]^{*}\to Lie^{k_1}(V[1]^{*})\wedge\dots\wedge Lie^{k_n}(V[1]^{*})
\end{eqnarray*}
Conjugate map:
\begin{eqnarray*}
m_{k_1,k_2, \dots ,k_n}:  V^{\otimes^{k_1}}\otimes \dots \otimes V^{\otimes^{k_n}}\to V.
\end{eqnarray*}
of degree $3-n-k_1-...-k_n$, satisfying bilinear relations. \\
\vspace*{3mm} 
In our previous notation $m_1=Q$, $m_{2}$-symmetrized LZ product, $m_{1,1}$--antisymmetrized LZ bracket.\\

\vspace*{3mm}

$L_{\infty}$ is generated 
by $m_1\equiv Q$, $m_{1,1,...,1}\equiv [\cdot ,\dots, \cdot ]$ and $C_{\infty}$ is generated by $m_1\equiv Q$, $m_k\equiv (\cdot, \dots, \cdot)$. \\
\vspace*{2mm} 
An important feature of 
$L_{\infty}$ algebra is a Maurer-Cartan equation ($\Phi$ is of degree 2) :
\begin{eqnarray*}
Q\Phi+\sum_{n\ge 2}\frac{1}{n!}[\underbrace{\Phi,\dots, \Phi}_{n}]+\dots=0,
\end{eqnarray*}
which has infinitesimal symmetries:
\begin{eqnarray*}
\Phi\to \Phi+Q\Lambda+\sum_{n\ge 1}\frac{1}{n!}[\underbrace{\Phi\dots\Phi}_{n}, \Lambda] 
\end{eqnarray*}


\end{frame}

\begin{frame}[t]
\frametitle{Quasiclassical limit of LZ $G_{\infty}$ algebra}
The following complex $(\mathcal{F}^{\cdot}, Q)$:
\begin{eqnarray*}
\xymatrixcolsep{30pt}
\xymatrixrowsep{3pt}
\xymatrix{
& \mathcal{V}\ar[ddddr] & h\mathcal{V}\ar[ddddr]^{\frac{1}{2}h{\rm div}}& \\
&& \p &&\\
& \bigoplus & \bigoplus & \\
&& {-\frac{1}{2}h{\rm div}} &&\\
\mathcal{O}_M\ar[uuuur]^{\p} & h\mathcal{O}_M\ar[uuuur]\ar[r]_{i\rm{d}} & h\mathcal{O}_M  & h^2\mathcal{O}_M
}
\end{eqnarray*}
is a subcomplex of 
$ (\mathcal{F}_h^{\cdot}, Q)$.
Then
\begin{eqnarray*}
&&(\cdot, \cdot)_h: \cF^i\otimes \cF^j\to \cF^{i+j}[h], 
\quad  \{\cdot,\cdot\}: \cF^i\otimes \cF^j\to h\cF_{i+j-1}[h],\\
&&{\bf b}: \cF^i\to h\cF^{i-1}[h], \nonumber
\end{eqnarray*}
so that 
\begin{eqnarray*}
(\cdot,\cdot)_0= \lim_{h\to 0}(\cdot,\cdot)_h, \quad 
\{\cdot,\cdot\}_0= \lim_{h\to 0}h^{-1}\{\cdot,\cdot\}_h, \quad \mathbf{b}_0=\lim_{h\to 0}h^{-1}\mathbf{b}
\end{eqnarray*}
are well defined. 


\end{frame}

%%%%%%%%%%%%%%%%%%%%%%%%%-------------15-------------%%%%%%%%%%%%%%%%%%%%%%%
\begin{frame}[t]

The symmetrized operations $(\cdot, \cdot)_0$, $\{\cdot,\cdot\}_0, \dots$ satisfy the relations of the homotopy Gerstenhaber algebra, so that all non-covariant higher-order terms disappear from the multilinear operations. \\

\vspace*{3mm}

The resulting $C_{\infty}$ and $L_{\infty}$ algebras are reduced to $C_3$ and $L_3$ algebras.  \\
\vspace*{3mm}

\hfill {\color{blue} \tiny{A.Z., Comm. Math. Phys. 303 (2011) 331-359.}}

\vspace*{5mm} \underline{Conjecture}: This $G_{\infty}$-algebra is the $G_3$-algebra (no homotopies beyond trilinear operations).

\vspace*{5mm}
Classical limit procedure for vertex algebroid (due to P. Bressler): $[\cdot, \cdot]_0=\lim_{h\to 0}\frac{1}{h}[\cdot, \cdot]$, $\pi_0=\lim_{h\to 0}\frac{1}{h}\pi$, 
$\langle\cdot, \cdot\rangle_0 =\lim_{h\to 0}\frac{1}{h}\langle\cdot, \cdot\rangle$.\\

\vspace{5mm}

The resulting operations form a Courant algebroid (Z.-J. Liu, A. Weinstein, P. Xu, 1997)

\end{frame}
\begin{frame}

A  Courant $\mathcal{O}_M$-algebroid is an $\mathcal{O}_M$-module $\cQ$
equipped with a structure of a Leibniz $\mathbb{C}$-algebra
$[ \cdot,\cdot ]_0 : \cQ\otimes_\mathbb{C}\cQ \to \cQ $, 
an $\mathcal{O}_M$-linear map of Leibniz algebras (the anchor map)
$
\pi_0 : \cQ \to \Gamma(TM)
$,
a symmetric $\mathcal{O}_M$-bilinear pairing
$\langle\cdot, \cdot\rangle: \cQ\otimes_{\mathcal{O}_M}\cQ \to \mathcal{O}_M $,  
a derivation
$
\p : \mathcal{O}_M \to \cQ
$
which satisfy
\begin{eqnarray*}
&&\pi\circ\partial =  0, \quad[q_1,fq_2]_0 = f[q_1,q_2] _0+ \pi_0(q_1)(f)q_2\\
&&\langle [q,q_1],q_2\rangle + \langle q_1,[q,q_2]\rangle  =  \pi_0(q)(\langle q_1, q_2\rangle_0), \\
&&[q,\partial(f)]_0  =  \partial(\pi_0(q)(f)) \nonumber\\
&&\langle q,\partial(f)\rangle  =  \pi_0(q)(f) \quad [q_1,q_2]_0 + [q_2,q_1]_0  =  \partial\langle q_1, q_2\rangle_0 \nonumber
\end{eqnarray*}
for $f\in\mathcal{O}_M$ and $q,q_1,q_2\in\cQ$.\\

First it was obtained as an analogue of Manin's double for Lie bialgebroid.

\vspace*{3mm}

 In our case $\cQ\cong\mathcal{O}(\cE)$, $\pi_0$ is just a projection on $\mathcal{O}(TM)$
\begin{eqnarray*}
[q_1,q_2]_0=-[q_1,q_2]_D, \quad \langle q_1, q_2\rangle_0=-\langle q_1, q_2\rangle^s, \quad \p=d.
\end{eqnarray*}

\end{frame}
\begin{frame}

The corresponding $L_3$-algebra on the half-complex  for Courant algebroid was constructed by D. Roytenberg and A. Weinstein (1998). 

\vspace*{3mm}

We show that it is a part of a more general structure, homotopy Gerstenhaber algebra.\\

\vspace*{3mm}

\underline{Question}: Is there a direct path (avoiding vertex algebra) from Courant algebroid to $G_3$-algebra? Odd analogue of Manin double?\\

\vspace*{3mm}

\underline{Remark}. $C_3$-algebra is  related to gauge theory. The appropraite "metric" deformation  gives a Yang-Mills $C_3$-algebra on a flat space.\\  
\hfill {\color{blue} \tiny{A.Z., Comm. Math. Phys. 303 (2011) 331-359.}}


\end{frame}

%%%%%%%%%%%%%%%%%%%%%%%%%%-----------16------------%%%%%%%%%%%%%%%%%%%%%%%%%%%%
\section[Einstein Equations]{Einstein Equations from $G_{\infty}$-algebras }

\begin{frame}[t]
%[t,label=current]
\frametitle{Simplest version: $G_{\infty}$ $\to$ Gerstenhaber algebra}
Subcomplex $(\mathcal{F}^{\cdot}_{sm}, Q)$:
\begin{eqnarray*}
\xymatrixcolsep{30pt}
\xymatrixrowsep{3pt}
\xymatrix{
& \mathcal{O}(T^{(1,0)}M)\ar[ddddr] & \mathcal{O}(T^{(1,0)}M)\ar[ddddr]^{\frac{1}{2}{\rm div}}& \\
&& 0 &&\\
& \bigoplus & \bigoplus & \\
&& {-\frac{1}{2}{\rm div}} &&\\
\mathbb{C}\ar[uuuur]^{0} & \mathbb{C}\ar[uuuur]\ar[r]_{i} & \mathcal{O}_M  & \mathcal{O}_M
}
\end{eqnarray*}
The $G_{\infty}$ algebra degenerates to G-algebra. Moreover, due to $\mathbf{b}_0$ it is a BV-algebra.
Combine chiral and antichiral part:
\begin{eqnarray*}
{\bf F}^{\cdot}_{sm}=\mathcal{F}^{\cdot}_{sm}{\otimes}\bar {\mathcal{F}}^{\cdot}_{sm} 
\end{eqnarray*}

\begin{eqnarray*}\label{brack2}
(-1)^{|a_1|}\{a_1,a_2\}=
\mathbf{b^-}(a_1,a_2)-(\mathbf{b^-}a_1,a_2)-(-1)^{|a_1|}(a_1\mathbf{b^-}a_2)\nonumber,
\end{eqnarray*}

where $\mathbf{b^-}=\mathbf{b}-\bar{\mathbf{b}}$.

\end{frame}

\begin{frame}
Maurer-Cartan elements, closed under $\mathbf{b}^-$: 
 $$
\Gamma(T^{(1,0)}(M)\otimes T^{(0,1)}(M))\oplus \mathcal{O}(T^{(0,1)}(M)\oplus \mathcal{O}(T^{(1,0)}(M)\oplus
\mathcal{O}_M\oplus \bar{\mathcal{O}}_M$$  

Components:$(g, \b v, v, \phi, \b \phi)$.\\

\vspace*{2mm}

The Maurer-Cartan equation is equivalent to:\\
\hfill {\color{blue} \tiny {A.Z., Nucl. Phys. B 794 (2008) 370-398; A.Z. ATMP, (2015), to appear.}}
\vspace{2mm}

1). Vector field $div_{\Omega}g$, where  $\log{\Omega}=-2\Phi_0=-2(\phi'+\b \phi'+\phi+\b \phi)$ and $\p_i{\p}_{\b j}\Phi_0=0$, is such that its $\Gamma(T^{(1,0)} M)$, $\Gamma(T^{(0,1)} M)$ components are correspondingly holomorphic and antiholomorphic.

2).  Bivector field $g\in \Gamma(T^{(1,0)}M\otimes T^{(0,1)}M)$ obeys the following equation: 
\begin{eqnarray*}\label{bil}
[[g,g]]+\mathcal{L}_{div_{\Omega}(g)}g=0, 
\end{eqnarray*} 
where $\mathcal{L}_{div_{\Omega}(g)}$ is a Lie derivative with respect to 
the corresponding vector fields and\\

\begin{eqnarray*}
[[g,h]]^{k\b l}\equiv
(g^{i\b j}\p_i\p_{\b j}h^{k\b l}+h^{i\b j}\p_i\p_{\b j}g^{k\b l}-\p_ig^{k\b j}\p_{\b j}h^{i\b l}-
\p_ih^{k\b j}\p_{\b j}g^{i\b l})\nonumber
\end{eqnarray*}

3). $div_{\Omega}div_{\Omega}(g)=0$.\\


\end{frame}

\begin{frame}
These are Einstein equations with the following constraints:
\begin{eqnarray*}\label{constr}
&&G_{i\bar{k}}=g_{i\bar{k}}, \quad B_{i\bar{k}}=-g_{i\bar{k}}, \quad \Phi=\log\sqrt{g}+\Phi_0,\\
&&G_{ik}=G_{\b i \b k}=G_{ik}=G_{\b i \b k}=0,\nonumber
\end{eqnarray*}

Physically:

\begin{eqnarray*}
&&\int [dp][d\b p][dX][d\b X]e^{-\frac{1}{2\pi i h}\int_\Sigma (\langle p\wedge\bar{\partial} X\rangle-
\langle \bar{p}\wedge{\partial} X\rangle
-\langle g, p\wedge \bar{p} \rangle)+\int_{\Sigma} R^{(2)}(\gamma)\Phi_0(X)}=\\
&&\int [dX][d\b X]e^{\frac{-1}{4\pi h}\int d^2 z
(G_{\mu\nu}+B_{\mu\nu})\partial X^{\mu}\bar{\partial}X^{\nu}+\int R^{(2)}(\gamma)(\Phi_0(X)+\log\sqrt{g})}
\end{eqnarray*}

based on computations of \\
A. Tseytlin and A. Schwarz, Nucl.Phys. B399 (1993) 691-708. 

\end{frame}

\begin{frame}
\frametitle{Main Conjecture}
Consider
\begin{eqnarray*}
{\bf F}^{\cdot}_{b^-}=\mathcal{F}^{\cdot}{\otimes}\bar {\mathcal{F}}^{\cdot}\vert_{b^-=0} 
\end{eqnarray*}

with the $L_{\infty}$-algebra structure given by Lian-Zuckerman construction. \\

One can explicitly check that GMC symmetry ($\Psi=\Psi(\mathbb{M}, \Phi, \rm{auxiliary~\ fields})$
\begin{eqnarray*}
\Psi\to \Psi+Q\Lambda+[\Psi,\Lambda]_h+\frac{1}{2}[\Psi,\Psi,\Lambda]_h+...,
\end{eqnarray*} 
reproduces
\begin{eqnarray*}
{\mathbb{M}}\to {\mathbb{M}}-D\alpha+ \phi_1(\alpha,{\mathbb{M}})+\phi_2(\alpha,{\mathbb{M}},{\mathbb{M}}).
\end{eqnarray*}

\vspace{5mm}

\underline{Conjecture}: The corresponding Maurer-Cartan equation gives Einstein equations on $G,B, \Phi$ expressed in terms of Beltrami-Courant differential. The symmetries of the Maurer-Cartan equation reproduce mentioned above symmetries of Einstein equations.
\end{frame}




\begin{frame}[t]
\vspace{35mm}
\begin{center}
\Huge\center{Thank you!}
\end{center}
\end{frame}

\end{document}
%%%%%%%%%%%%%%%%%%%%%%%%%%%%%%%%%%%%%%%%%%%%%%%%%%%%%%%%%%%%%%%%%%%%%%%%%%%%%%%%%%
